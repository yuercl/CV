%% start of file `template-zh.tex'.
%% Copyright 2006-2013 Xavier Danaux (xdanaux@gmail.com).
%
% This work may be distributed and/or modified under the
% conditions of the LaTeX Project Public License version 1.3c,
% available at http://www.latex-project.org/lppl/.


\documentclass[11pt,a4paper,kai]{moderncv}   % possible options include font size ('10pt', '11pt' and '12pt'), paper size ('a4paper', 'letterpaper', 'a5paper', 'legalpaper', 'executivepaper' and 'landscape') and font family ('sans' and 'roman')

% moderncv 主题
\moderncvstyle{banking}                        % 选项参数是 ‘casual’, ‘classic’, ‘oldstyle’ 和 ’banking’
\moderncvcolor{red}                          % 选项参数是 ‘blue’ (默认)、‘orange’、‘green’、‘red’、‘purple’ 和 ‘grey’
%\definecolor{color0}{rgb}{0,0,0}% black
%\definecolor{color1}{rgb}{0.95,0.20,0.20}% red
%\definecolor{color2}{rgb}{0.0,0.0,0.0}% dark grey
%\nopagenumbers{}                             % 消除注释以取消自动页码生成功能

% 字符编码
\usepackage[utf8]{inputenc}                   % 替换你正在使用的编码
\usepackage{CJKutf8}
%\usepackage{hyperref}                         % 用于添加超链接

% 调整页面出血
\usepackage[scale=0.75]{geometry}
%\setlength{\hintscolumnwidth}{3cm}           % 如果你希望改变日期栏的宽度

% 个人信息
\name{陈}{龙}
%\title{简历题目 (可选项)}                     % 可选项、如不需要可删除本行
\address{天府软件园(成都)}{邮编:610041}{}           % 可选项、如不需要可删除本行
\phone[mobile]{(+86)~136~4806~2548}              % 可选项、如不需要可删除本行
%\phone[fixed]{+2~(345)~678~901}               % 可选项、如不需要可删除本行
%\phone[fax]{+3~(456)~789~012}                 % 可选项、如不需要可删除本行
\email{yuerguang.cl@gmail.com}                    % 可选项、如不需要可删除本行
\homepage{yuercl.github.io}                  % 可选项、如不需要可删除本行
\extrainfo{擅长语言(Java、Bash、PHP、SQL、C)}                 % 可选项、如不需要可删除本行
%\photo[64pt][0.4pt]{picture}                  % ‘64pt’是图片必须压缩至的高度、‘0.4pt‘是图片边框的宽度 (如不需要可调节至0pt)、’picture‘ 是图片文件的名字;可选项、如不需要可删除本行
\quote{乐于学习,勇于挑战自我}                          % 可选项、如不需要可删除本行

% 显示索引号;仅用于在简历中使用了引言
%\makeatletter
%\renewcommand*{\bibliographyitemlabel}{\@biblabel{\arabic{enumiv}}}
%\makeatother

% 分类索引
%\usepackage{multibib}
%\newcites{book,misc}{{Books},{Others}}
%----------------------------------------------------------------------------------
%            内容
%----------------------------------------------------------------------------------
\begin{document}
%\begin{CJK}{UTF8}{gbsn}                       % 详情参阅CJK文件包
\begin{CJK}{UTF8}{kai}                       % 详情参阅CJK文件包
\maketitle

\section{个人信息}
\cvdoubleitem{姓名}{陈龙}{性别}{男}
\cvdoubleitem{年龄}{23(1991-02)}{现居地}{四川成都}
\cvdoubleitem{毕业时间}{2013-6}{民族}{汉族}

\section{教育背景}
\cventry{2009年 -- 2013年}{数学与应用数学}{电子科技大学}{成都}{\textit{理学学士}}{2010年获三等奖学金}  % 第3到第6编码可留白

%\section{毕业论文}
%\cvitem{CRM}{\emph{题目}}
%\cvitem{导师}{导师}
%\cvitem{说明}{\small 论文简介}

\section{工作背景}
\subsection{正式工作}
\cventry{2013年3月 -- 2014年至今}{软件研发工程师}{成都神州通付科技有限公司}{成都}{(领域:移动互联网O2O)}{%
工作内容:%
\begin{itemize}%
    \item App后端以及后台系统协议制定与实现
    \begin{itemize}%
          \item 设计框架
            \begin{itemize}
                \item Java,DbUtils,Java NIO
                \item 由消息中间件协调后端程序处理业务逻辑
            \end{itemize}
      \end{itemize}
    \item 编写后台管理系统
    \begin{itemize}%
          \item 设计框架
            \begin{itemize}
                \item PHP,Jquery,EasyUI,etc...
                \item PHP链接消息中间件请求后端业务程序
            \end{itemize}
      \end{itemize}
    \item 为项目提供数据支持,如抓取数据
    \item 车牌识别研究(基于OpenCV)
\end{itemize}}
\subsection{实习全职}
\cventry{2012年7月 -- 2013年2月}{软件研发工程师}{北京知行九易科技有限公司}{北京\&成都}{(公司规模20人,开发共4人,领域:外贸电子商务)}{%
工作内容:%
\begin{itemize}%
    \item 电子商务供应链模块开发:
          \begin{itemize}%
              \item 设计框架
                \begin{itemize}
                    \item J2EE(SSH),EasyUI,jQuery,Maven,MySQL,PowerDesigner
                    \item MQ消息中间件(ActiveMQ),工作流引擎,规则引擎
%                    \item 项目采用敏捷开发方式
                \end{itemize}
              \item 持续时间,六个月
              \item 完成情况
                \begin{itemize}
                    \item 开发共三人,于3m*3m会议室内开发,完成供应链管理系统中采购相关流程,竞拍采购原型,部分前端页面编码,同时与订单系统、仓库系统和商品库系统对接
                \end{itemize}
          \end{itemize}
\end{itemize}}

\section{参与项目}
\subsection{科技新闻抓取}
科技新闻抓取和苹果新闻提取。该程序基于SpringMVC、Spring Crontab、JSoup \& HttpClient 、Shiro 、Jetty和Bootstrap等框架开发,目前已实现提取新闻和期发帖到到指定论坛。后续会加上分类提取算法(仅提取苹果相关新闻)和全文搜索功能.在此之前利用Shell(Bash)实现登陆论坛发帖功能和Keep Online的小工具
\subsection{Tools4You}
 Tools4You是根据\href{https://github.com/Mailerm/Tools4ME}{Tools4ME}为原型而来,用于提供项目内部使用的常用工具集合的Java桌面程序,基于Swing图像界面开发,目前已实现功能有Base64、Unicode、UrlEncode、MD5 等加解密和与消息中间件通信的调试插件
\subsection{Web安全研究}
 由于在校期间的开发的\href{http://www.math.uestc.edu.cn}{数学学院官网网站}被黑客入侵,遂开始研究Web安全和渗透相关知识,对常见Web安全问题如:SQL注入、XSS、XSRF等有了解,后因开发任务集中而较少关注安全,曾提交通达OA中邮件存在持久性XSS漏洞,可攻击收件人(WooYun-2013-25365) 等漏洞
%\cvlistitem{其他}
\subsection{WSO2中间件平台研究}
于2012年暑假直到2013年12月跟随老师,学习\href{http://www.wso2.com}{WSO2}中间件平台,完成基于WSO2中间件平台的人力资源管理项目演示版,涉及WSO2中间件平台的Enterprise Service Bus,Data Service Bus 等

\section{专业技能}
\subsection{语言技能}
%\cvitemwithcomment{全国大学英语四级考试}{通过}{}
\cvitemwithcomment{\textsf{全国大学英语四级考试}}{通过,能正常阅读英文文档}{}
%\cvitemwithcomment{语言 2}{水平}{评价}
%\cvitemwithcomment{语言 3}{水平}{评价}

\subsection{计算机技能}
%\cvdoubleitem{团队开发}{XXX, YYY, ZZZ}{类别 5}{XXX, YYY, ZZZ}
\cvitem{操作系统}{Linux(Ubuntu\&CentOS), Windows}{}{}
\cvitem{团队开发}{Subversion, Git(for self)}{}{}
\cvitem{编程语言}{Java, Shell, PHP}{}{}
\cvitem{开发工具}{Eclipse, Vim, IntelliJ IDEA}{}{}
\cvitem{其他工具}{Cygwin, Markdown, \LaTeX}{}{}

\subsection{其它技能}
\cvitem{\textsf{学习能力}}{对新技术有浓厚的兴趣,有很强的自学能力}
\cvitem{\textsf{团队合作}}{拥有良好的沟通能力,有很强的团队意识}

%\subsection{项目}
%\cvlistitem{项目项目项目}
%\cvlistitem{项目 2}
%\cvlistitem{项目 3}

\renewcommand{\listitemsymbol}{-}             % 改变列表符号

%\section{其他 2}
%\cvlistdoubleitem{项目 1}{项目 4}
%\cvlistdoubleitem{项目 2}{项目 5\cite{book1}}
%\cvlistdoubleitem{项目 3}{}

% 来自BibTeX文件但不使用multibib包的出版物
%\renewcommand*{\bibliographyitemlabel}{\@biblabel{\arabic{enumiv}}}% BibTeX的数字标签
%\nocite{*}
%\bibliographystyle{plain}
%\bibliography{publications}                    % 'publications' 是BibTeX文件的文件名

% 来自BibTeX文件并使用multibib包的出版物
%\section{出版物}
%\nocitebook{book1,book2}
%\bibliographystylebook{plain}
%\bibliographybook{publications}               % 'publications' 是BibTeX文件的文件名
%\nocitemisc{misc1,misc2,misc3}
%\bibliographystylemisc{plain}
%\bibliographymisc{publications}               % 'publications' 是BibTeX文件的文件名

\clearpage\end{CJK}
\end{document}


%% 文件结尾 `template-zh.tex'.
